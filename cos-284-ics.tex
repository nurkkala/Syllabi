\documentclass[11pt]{article}

\usepackage{framed}
\usepackage{xcolor}

\newcommand{\syllabuslayout}[3]{
  \usepackage[margin=1.5in]{geometry}
  \usepackage{lastpage}

  \usepackage{fancyhdr}
  \renewcommand{\headrulewidth}{0.4pt}
  \renewcommand{\footrulewidth}{0.4pt}
  \pagestyle{fancy}

  \lhead{Course Syllabus}
  \chead{}
  \rhead{Page \thepage\ of \pageref{LastPage}}
  \lfoot{#1}
  \cfoot{#2}
  \rfoot{#3~\the\year}
}

\newcommand{\titleblock}[2]{
  \begin{center}
    \color{purple}
    {\bfseries
      {\Large #1}\\[4pt]
      {\Large #2}}\\[12pt]
    Department of Computer Science and Engineering\\
    Taylor University
  \end{center}
}

\newcommand{\logistics}[4]{
  \begin{center}
    #1 Credit Hours\\
    #2 $\bullet$\ #3\\[1em]
    Final Exam: #4
  \end{center}
}

\newcommand{\lastupdated}{
\vfill
\begin{flushright}
  \footnotesize
  \textit{This document last updated \today}.
\end{flushright}}

\newenvironment{catalogentry}[1]
{\section{Description}
  \textbf{Prerequisites}: #1\\

  \noindent
  \textbf{From the catalog}:}
{}

\newcommand{\safari}{O'Reilly Learning}
\syllabuslayout{COS~284}{Introduction to Computer Systems}{Summer}

\usepackage[colorlinks=true,allcolors=blue]{hyperref}
\usepackage{booktabs}
\usepackage{comment}

\usepackage{biblatex}
\addbibresource{courses.bib}

\begin{document}

\section{Instructor}

\begin{center}
  \begin{tabular}{rll}
    \toprule
    \multicolumn{3}{l}{\textbf{Dr.\ Tom Nurkkala}}                            \\
    \multicolumn{3}{l}{Associate Professor, Computer Science and Engineering} \\
    \multicolumn{3}{l}{Director, Center for Missions Computing}               \\
    \midrule
    Office & \multicolumn{2}{l}{Euler Science Complex 211}                    \\
    Email  & \multicolumn{2}{l}{\texttt{tnurkkala@cse.taylor.edu}}            \\
    Phone  & \multicolumn{2}{l}{765/998-5163}                                 \\
    Hours  & Mon, Wed & 2:00--3:00                                            \\
           & Tue, Thu & 10:00--12:00                                          \\
           & \multicolumn{2}{l}{\emph{Or by appointment}}                     \\
    \bottomrule
  \end{tabular}
\end{center}

%%% Local Variables:
%%% mode: latex
%%% End:

%  LocalWords:  rrl


\section{Credit}

This course is based on the Introduction to Computer Systems course
at Carnegie Mellon University (CMU).
Professors Randal Bryant and David O’Hallaron,
who developed the course and wrote the textbook (Section 5),
have generously made available course material
(including notes, syllabi, laboratory exercises, and lecture slides)
for use at other institutions.
I make considerable use of their excellent materials throughout the course
(including in this document!) and wish to credit and thank them.

I have also drawn liberally from previous offerings of this course
at Taylor University,
and wish to thank Professor Jonathan Geisler for his kind assistance.

\section{Course Overview}

The aim of the course is to help you become a better programmer by teaching you
the basic concepts underlying all computer systems. I want you to learn what
really happens when your programs run, so that when things go wrong (as they
always do) you will have the intellectual tools to solve the problem.

Why do you need to understand computer systems if you do all of your
programming in high-level languages? In most of computer science, we're pushed
to make abstractions and stay within their frameworks. But, any abstraction
ignores effects that can become critical. As an analogy, Newtonian mechanics
ignores relativistic effects. The Newtonian abstraction is completely
appropriate for bodies moving at less than \(0.1c\), but higher speeds require
working at a greater level of detail.

The following ``realities'' are some of the major areas where the abstractions
you've learned in previous classes break down:
\begin{enumerate}
\item \textbf{An \texttt{int} isn't an integer; a \texttt{float} isn't a real}.
  Our finite representations of numbers have significant limitations,
  and because of these limitations
  we sometimes have to think in terms of bit-level representations.
\item \textbf{You've got to know assembly language}.
  Even if you never write programs in assembly,
  the behavior of a program sometimes cannot be understood
  based purely on the abstraction of a high-level language.
  Furthermore, understanding the effects of bugs
  requires familiarity with the machine-level model.
\item \textbf{Memory matters}.
  Computer memory is not unbounded.
  It must be allocated and managed.
  Memory referencing errors are especially pernicious.
  An erroneous updating of one object can cause a change
  in some logically unrelated object.
  Also, the combination of caching and virtual memory
  provides the functionality of a uniform unbounded address space,
  but not the performance.
\item \textbf{There is more to performance than asymptotic complexity}.
  Constant factors also matter.
  There are systematic ways to evaluate and improve program performance.
\item \textbf{Computers do more than execute instructions}.
  They also need to get data in and out
  and they interact with other systems over networks.
\end{enumerate}

By the end of the course, you will understand these realities in some detail.
As a result, you will have learned skills and knowledge
that will help you throughout your education and career as a computer scientist.

\section{Learning Objectives}

By the end of the semester, you should be able to:

\begin{enumerate}
\item Describe how a computer represents integers internally.
\item Convert twos complement binary values to decimal.
\item Convert binary to decimal and hexadecimal.
\item Understand how a computer represents fractional values.
\item Convert a fractional decimal value to its IEEE single- or
  double-precision representation.
\item Read x86 assembly code.
\item Write simple x86 assembly functions.
\item Describe how a computer executes a function call including parameters,
  return values, and the stack.
\item Describe how compilers convert C code to x86 assembly.
\item Explain how various C data structures are accessed in x86 assembly.
\item Describe the motivation for a memory hierarchy.
\item Explain how a memory hierarchy works.
\item Understand how a cache is organized internally.
\item Create programs that take advantage of the memory hierarchy.
\item Explain how a linker works.
\item Contrast static and dynamic linking.
\item Explain what a system call is and how system calls work.
\item Explain the fork system call.
\item Explain the exec family of system calls.
\item Describe how hardware and the operating system cooperate to deal with
  exceptional situations.
\item Measure the performance of an application accurately.
\item Improve the performance of an application based on good measurement
  techniques.
\item Understand virtual memory and how it works.
\item Explain the benefits of virtual memory.
\item Explain the role of the translation look-aside buffer.
\item Describe the virtual memory organization of a process running with Linux
  on an x86 processor.
\item Explain how a dynamic memory allocator works.
\item Be familiar with garbage collection.
\item Be familiar with common memory bugs and how to avoid them.
\item Explain what concurrent computation is and how to achieve it.
\item Contrast process-based concurrency with I/O multiplexing-based
  concurrency.
\item Describe why synchronization is necessary.
\item Be familiar with how to achieve synchronization.
\end{enumerate}

\section{Texts}

The text by Bryant and O'Halaron~\cite{cs:app3} is required.
We will refer to it as \texttt{CS:APP}.
You are \emph{encouraged} to purchase---or have readily available---the
updated edition of ``K\&R,'' the classic reference on C.\cite{k&r}

It is \emph{very important} that you read carefully
the assigned readings in the \texttt{CS:APP} book.
You will get the most benefit if you read assigned passages
\emph{before} they are covered in class.
A detailed guide to reading assignments can be found in the course schedule.

The textbook includes many practice problems throughout the body of each
chapter. These are straightforward exercises that help you understand the
material you have just read by using it \emph{immediately}. For self-study,
solutions for all practice problems appear at the end of each chapter.

In short, the best way to use the textbook to enhance your learning is as
follows:
\begin{enumerate}
\item Read the assigned readings \emph{before} the corresponding class.
\item Work the practice problems \emph{immediately} as you encounter them in
  the text.
\item Check your work—and your understanding—with the answer key.
\end{enumerate}

\section{Topics}

These are the topics that could potentially be covered in the course.
It is unlikely that we will have time to cover all this material.
\begin{itemize}
\item Course Overview
\item Bits and Bytes
  \begin{itemize}
  \item Basics
  \item Integers
  \item Floating Point
  \end{itemize}
\item Machine Programming
  \begin{itemize}
  \item Basics
  \item Control
  \item Procedures
  \item Data
  \item Advanced
  \end{itemize}
\item Program Optimization
\item Memory Hierarchy
\item Cache Memory
\item Linking
\item Exceptional Control Flow
  \begin{itemize}
  \item Exceptions and Processes
  \item Signals and Non-Local Jumps
  \end{itemize}
\item Virtual Memory
  \begin{itemize}
  \item Concepts
  \item Systems
  \end{itemize}
\item Dynamic Memory
  \begin{itemize}
  \item Basic
  \item Advanced
  \end{itemize}
\item Internetworking
  \begin{itemize}
  \item Network Programming
  \item Web Servers
  \end{itemize}
\item Concurrent Programming
  \begin{itemize}
  \item Sychronization
  \end{itemize}
\end{itemize}

\section{Labs}

The labs come directly from the textbook authors.
They have a very nice system set up so that you can test and submit
all your programs online and will
already know how you are doing on the assignment prior to submission.
This should make it very easy to know when you have a correct solution
and when you need to keep working.

Please do not look for solutions online.
Doing so constitutes cheating.
The authors have introduced randomness into the assignments
so solutions posted by others may be wrong anyway.

You will use Linux for all lab work.
For some labs, you will be given a binary
that works on the CSE machines,
but should probably work on any recent vintage of Linux.
For others, you are required to develop code for a Linux box.
You may use any machine to develop,
but \emph{it must run correctly on the CSE machines}.
You must check your work on the machines in the laboratories
before submitting it.

Most low-level systems code is written in~C,
and you will be required to do the same.
You will also be required to read, understand, generate,
or otherwise fiddle with x86 assembly on a Linux box.

\section{Evaluation}

The grading scheme for the course will be announced in the next few days.

\section{Expectations}

Following are my expectations regarding the course.

\subsection{Attendance}
\label{sec:attendance}

You are required to attend all class sessions.
I will be in class each day, and I expect you to be there also.

In general, I am very understanding about students who must miss class
due to a sanctioned Taylor activity, medical appointment, job interview,
family emergency, and the like.
If possible, let me know in advance that you will not be in class;
I will work with you to arrange make-up instruction, homework, exams, etc.

\subsection{Late Work}

All course assignments will include an unambiguous due date.
Usually, assignments are due at the beginning of class on the due date.
If there are multiple sections of a class,
the assignment is due at the beginning of the earliest such section.
Barring exceptional circumstances like those mentioned in section~\ref{sec:attendance},
I expect your work to be submitted \emph{on the due date}.
Late work will \emph{not} be accepted.

This policy on late work is intended to prepare you
for real-world experience after graduation.
In the marketplace,
late work is not merely an inconvenience.
Missing a deadline may
alienate your customer,
upset your manager,
ruin your project,
or terminate your employment!
\emph{Now} is the time to learn the self discipline and time management skills
required to complete your work when it is due.

\subsection{Conduct}

I expect you to be prepared, awake, aware, and participatory during class. I will not
hesitate to ask you to stand or move if you are distracted or sleepy.

I expect you to join in discussions, respond to questions from me and from your
colleagues, and ask questions of me. I expect you to hold my feet to the fire if I am
being unclear, unkind, or contradictory.

\subsection{Gizmos}

You may not use a laptop, tablet, or similar device to check e-mail, engage in social
networking, surf the web, or any other activity not directly relevant
to current classroom activity.
If you use an electronic gizmo during class for legitimate academic purposes
(e.g., note taking), be prepared to demonstrate relevant use on demand
at any time.

\section{Moodle}

The Computer Science and Engineering department uses Moodle
as our Learning Management System.
The URL for Moodle is \url{https://moodle.cse.taylor.edu}.
To sign on to the course site for the first time,
you will need an enrollment key.
The key for this course is
\texttt{nerds4christ}.

You are responsible for checking Moodle regularly
to keep up with assignment due dates and other announcements.
For due dates, \emph{the Moodle calendar is your friend}.

\section{Slack}

This course will use Slack
for informal communication,
Q\&A,
last minute announcements,
jokes,
and the like.
You are \emph{strongly} encouraged to join the conversation.

Find the \emph{TU CSE Student} slack team at
\url{tucsestudents.slack.com}.
Look there for a \emph{channel}
dedicated to the course.

\section{Academic Integrity}

As a student at an institution whose goal is to honor Christ in all that it does,
I expect you to uphold the strictest standards of academic integrity.
You must do your own work,
cite others when you present their work,
and never misrepresent your academic performance in any way.
Violation of these standards stains the reputations of you as a student,
Taylor as an institution,
and Jesus as our Lord.

Every assignment should indicate clearly
that it is either:
\begin{itemize}
\item An \textbf{individual} assignment,
  to be done \emph{entirely by you},
  without any direct participation from other students.
\item A \textbf{group} assignment, to be done \emph{collectively with a group}
\end{itemize}
Unless otherwise stated,
assignments are \textbf{individual} assignments.

\begin{framed}
Note that you are \emph{always} welcome
to get help from the instructor.
\end{framed}

A violation of academic integrity may result in your failing the course
and other disciplinary action by the University.
Refer to the Taylor catalog for the official statement of these ideas.

%  LocalWords:  christ

%%% Local Variables:
%%% mode: latex
%%% TeX-master: "sys394-isd"
%%% End:


\section{Academic Integrity}

\section{Academic Integrity}

As a student at an institution whose goal is to honor Christ in all that it does,
I expect you to uphold the strictest standards of academic integrity.
You must do your own work,
cite others when you present their work,
and never misrepresent your academic performance in any way.
Violation of these standards stains the reputations of you as a student,
Taylor as an institution,
and Jesus as our Lord.

Every assignment should indicate clearly
that it is either:
\begin{itemize}
\item An \textbf{individual} assignment,
  to be done \emph{entirely by you},
  without any direct participation from other students.
\item A \textbf{group} assignment, to be done \emph{collectively with a group}
\end{itemize}
Unless otherwise stated,
assignments are \textbf{individual} assignments.

\begin{flushleft}
  \begin{framed}
    You are \emph{always} welcome
    to get help from the instructor on \emph{any}
    homework assignment or project,
    whether an individual or group assignment.
  \end{framed}
\end{flushleft}

\subsection{What Constitutes Academic Dishonesty?}
\label{sec:what-is}

For purposes of this course, the following are \emph{non-exhaustive} examples
of violations of academic integrity.
\begin{enumerate}
\item
  Sharing code or other electronic files by copying, retyping, looking at,
  or supplying a copy of a file from this or a previous semester. 
\item
  Sharing written assignments or exams by looking at, copying, or supplying
  an assignment or exam.
\item
  Using another student's code. Using code from this or previous offerings of the
  class, from courses at other institutions, or from any other source (e.g.,
  software found on the Internet).
\item\label{i:looking}
  Looking at another student's code. Although mentioned above, it bears
  repeating: looking at other students' code or allowing others to look at yours
  is academic dishonesty. There is no notion of looking ``too much,'' since no looking is
  allowed at all.
\end{enumerate}

\subsection{What Does Not Constitute Academic Dishonesty?}

In contrast, the following are \emph{non-exhaustive} examples of activities
that \emph{do not} violate academic integrity.

\begin{enumerate}
\item Clarifying ambiguities or vague points in class handouts or textbooks.
\item Helping others use the computer systems, networks, compilers, debuggers,
  profilers, or other system facilities without regard to a particular assignment or project.
\item Helping others with high-level design issues.
\item Helping others with high-level (\emph{not} code-based) debugging.
\item Using code provided by the instructor from the course web site or elsewhere.
\end{enumerate}

\subsection{From the Provost}

Taylor's Provost\footnote{At Taylor, the \emph{Provost} is our Chief Academic Officer.}
defines \emph{plagiarism} as follows:
\begin{quote}
  In an instructional setting,
  plagiarism occurs when a person presents or turns in work
  that includes someone else's ideas, language,
  or other (not common-knowledge\footnote{\emph{Common knowledge}
    means any knowledge or facts that could be found in multiple places
    or as defined by a discipline, department, or faculty member.}) material
  without giving appropriate credit to the source.
  Plagiarism will not be tolerated
  and may result in failing this course,
  and may also result in further consequences
  as stipulated in the
  \href{http://www.taylor.edu/academics/files/undergrad-catalog/current/catalog.pdf}{Taylor catalog}.
\end{quote}

The Provost goes on to say:
\begin{quote}
  Academic dishonesty constitutes a serious violation
  of academic integrity and scholarship standards at Taylor
  that can result in substantial penalties,
  at the sole discretion of the University,
  including but not limited to,
  denial of credit in a course as well as dismissal from the University.

  In short, a student violates academic integrity
  when he or she claims credit for any work not his or her own
  (words, ideas, answers, data, program codes, music, etc.)
  or when a student misrepresents any academic performance.
  Please see 
  \href{http://www.taylor.edu/academics/files/undergrad-catalog/current/catalog.pdf}%
  {the catalog} for a complete statement.
\end{quote}

\subsection{Personal Reflection}

I wrote an initial version of this section
as a Slack message
in the wake of several students who admitted to cribbing code
for homework assignments
from the Internet:
\begin{quotation}
  \textbf{Plagiarism}---We all look to the Internet
  to search for quick solutions to all manner of computing problems:
  checking a function signature,
  getting more detail on the meaning of an error code,
  recalling the behavior of a language construct,
  or finding a canonical implementation of a standard algorithm.

  Professional honesty and integrity demand
  that if we employ the results of such a search in our own work,
  we do so only if:
  \begin{enumerate}
  \item
    It is permitted (e.g., if the material is posted publicly)
  \item 
    We cite the source (e.g., a comment in our code).
  \end{enumerate}
  This is true whether we're working on a project in college
  or on a multi-year product after we've been in professional practice for decades.
  
  Searching for documentation on a function,
  however,
  is \emph{qualitatively different}
  from searching for the solution to a homework assignment.
  The former search is a quick and effective way
  to shore up our understanding.
  The latter is the gateway to academic dishonesty: cheating by plagiarism.
  
  As mentioned (Section~\ref{sec:what-is}, item \ref{i:looking}),
  even \emph{looking} at code written by someone else
  as you try to solve a class assignment is expressly forbidden.
  The reason for this prohibition is simple:
  if you find someone else's solution to a homework assignment,
  you are you longer even \emph{able} to solve the problem independently
  (or with your team).
  You can't ``un-see'' the existing solution.
  Your solution can no longer be your own---it's your repackaging of someone else's work.
  You are now faced with three options:
  \begin{enumerate}
  \item
    Don't submit a solution at all
    because any solution you construct
    would not be your own work.
    You will receive no credit.
  \item
    Submit the work with an explicit citation
    of the source from which it is derived.
    You avoid the charge of plagiarism,
    but you will also receive no credit for the work
    because it's not yours.  
  \item
    Submit the work as if it is your own
    without citing the source.
    This is \emph{outright plagiarism}.
    According to university and course policy,
    it could result in penalties ranging from
    a zero on the assignment,
    to failing the course,
    to expulsion from the university.
  \end{enumerate}
  In other words,
  once you decide to search for answers for a homework assignment,
  if you want to be honest with yourself and others,
  you really have no good options.

  You might try to skirt this fact by telling yourself:
  This course is \emph{really} hard.
  The answers are out there \emph{somewhere} on the internet anyway.
  Nobody is really harmed by this ``one'' infraction.
  I can pretty up someone else's code and pass it off as my own.
  I can put one over on my prof or TA.
  And so forth.

  What you ought to do, however, is ask yourself:
  Do I want to be the kind of person who cheats?
  Am I willing to sell my professional birthright for a quick workaround?
  If I fail to act morally in this little thing,
  how can I expect to be entrusted with big things down the road?
  How will I feel when I'm fired from my job or sued for professional malpractice?

  I know plagiarism is happening.
  I sympathize with the challenges posed by our curriculum.
  I was a student for \emph{way} longer than you
  and faced similar pressure.
  What I'd ask are two things:
  \begin{enumerate}
  \item
    If you have committed plagiarism,
    admit to it,
    take a zero on that work,
    and clear your conscience.
  \item
    Go and sin no more.
    Stop cheating.
    Do your own work.
    Be proud of what you yourself can accomplish
    with the intellectual gifts God gave you.
  \end{enumerate}
\end{quotation}


%%% Local Variables:
%%% mode: latex
%%% End:

% LocalWords:  profilers else's


\printbibliography{}

\end{document}

%%% Local Variables:
%%% mode: latex
%%% TeX-master: t
%%% End:
