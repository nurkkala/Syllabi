\section{Academic Integrity}

As a student at an institution whose goal is to honor Christ in all that it does,
I expect you to uphold the strictest standards of academic integrity.
You must do your own work,
cite others when you present their work,
and never misrepresent your academic performance in any way.
Violation of these standards stains the reputations of you as a student,
Taylor as an institution,
and Jesus as our Lord.

Every assignment should indicate clearly
that it is either:
\begin{itemize}
\item An \textbf{individual} assignment,
  to be done \emph{entirely by you},
  without any direct participation from other students.
\item A \textbf{group} assignment, to be done \emph{collectively with a group}
\end{itemize}
Unless otherwise stated,
assignments are \textbf{individual} assignments.

\begin{flushleft}
  \begin{framed}
    You are \emph{always} welcome
    to get help from the instructor on \emph{any}
    homework assignment or project,
    whether an individual or group assignment.
  \end{framed}
\end{flushleft}

\subsection{What Constitutes Academic Dishonesty?}
\label{sec:what-is}

For purposes of this course, the following are \emph{non-exhaustive} examples
of violations of academic integrity.
\begin{enumerate}
\item
  Sharing code or other electronic files by copying, retyping, looking at,
  or supplying a copy of a file from this or a previous semester. 
\item
  Sharing written assignments or exams by looking at, copying, or supplying
  an assignment or exam.
\item
  Using another student's code. Using code from this or previous offerings of the
  class, from courses at other institutions, or from any other source (e.g.,
  software found on the Internet).
\item\label{i:looking}
  Looking at another student's code. Although mentioned above, it bears
  repeating: looking at other students' code or allowing others to look at yours
  is academic dishonesty. There is no notion of looking ``too much,'' since no looking is
  allowed at all.
\end{enumerate}

\subsection{What Does Not Constitute Academic Dishonesty?}

In contrast, the following are \emph{non-exhaustive} examples of activities
that \emph{do not} violate academic integrity.

\begin{enumerate}
\item Clarifying ambiguities or vague points in class handouts or textbooks.
\item Helping others use the computer systems, networks, compilers, debuggers,
  profilers, or other system facilities without regard to a particular assignment or project.
\item Helping others with high-level design issues.
\item Helping others with high-level (\emph{not} code-based) debugging.
\item Using code provided by the instructor from the course web site or elsewhere.
\end{enumerate}

\subsection{From the Provost}

Taylor's Provost\footnote{At Taylor, the \emph{Provost} is our Chief Academic Officer.}
defines \emph{plagiarism} as follows:
\begin{quote}
  In an instructional setting,
  plagiarism occurs when a person presents or turns in work
  that includes someone else's ideas, language,
  or other (not common-knowledge\footnote{\emph{Common knowledge}
    means any knowledge or facts that could be found in multiple places
    or as defined by a discipline, department, or faculty member.}) material
  without giving appropriate credit to the source.
  Plagiarism will not be tolerated
  and may result in failing this course,
  and may also result in further consequences
  as stipulated in the
  \href{http://www.taylor.edu/academics/files/undergrad-catalog/current/catalog.pdf}{Taylor catalog}.
\end{quote}

The Provost goes on to say:
\begin{quote}
  Academic dishonesty constitutes a serious violation
  of academic integrity and scholarship standards at Taylor
  that can result in substantial penalties,
  at the sole discretion of the University,
  including but not limited to,
  denial of credit in a course as well as dismissal from the University.

  In short, a student violates academic integrity
  when he or she claims credit for any work not his or her own
  (words, ideas, answers, data, program codes, music, etc.)
  or when a student misrepresents any academic performance.
  Please see 
  \href{http://www.taylor.edu/academics/files/undergrad-catalog/current/catalog.pdf}%
  {the catalog} for a complete statement.
\end{quote}

\subsection{Personal Reflection}

I wrote an initial version of this section
as a Slack message
in the wake of several students who admitted to cribbing code
for homework assignments
from the Internet:
\begin{quotation}
  \textbf{Plagiarism}---We all look to the Internet
  to search for quick solutions to all manner of computing problems:
  checking a function signature,
  getting more detail on the meaning of an error code,
  recalling the behavior of a language construct,
  or finding a canonical implementation of a standard algorithm.

  Professional honesty and integrity demand
  that if we employ the results of such a search in our own work,
  we do so only if:
  \begin{enumerate}
  \item
    It is permitted (e.g., if the material is posted publicly)
  \item 
    We cite the source (e.g., a comment in our code).
  \end{enumerate}
  This is true whether we're working on a project in college
  or on a multi-year product after we've been in professional practice for decades.
  
  Searching for documentation on a function,
  however,
  is \emph{qualitatively different}
  from searching for the solution to a homework assignment.
  The former search is a quick and effective way
  to shore up our understanding.
  The latter is the gateway to academic dishonesty: cheating by plagiarism.
  
  As mentioned (Section~\ref{sec:what-is}, item \ref{i:looking}),
  even \emph{looking} at code written by someone else
  as you try to solve a class assignment is expressly forbidden.
  The reason for this prohibition is simple:
  if you find someone else's solution to a homework assignment,
  you are you longer even \emph{able} to solve the problem independently
  (or with your team).
  You can't ``un-see'' the existing solution.
  Your solution can no longer be your own---it's your repackaging of someone else's work.
  You are now faced with three options:
  \begin{enumerate}
  \item
    Don't submit a solution at all
    because any solution you construct
    would not be your own work.
    You will receive no credit.
  \item
    Submit the work with an explicit citation
    of the source from which it is derived.
    You avoid the charge of plagiarism,
    but you will also receive no credit for the work
    because it's not yours.  
  \item
    Submit the work as if it is your own
    without citing the source.
    This is \emph{outright plagiarism}.
    According to university and course policy,
    it could result in penalties ranging from
    a zero on the assignment,
    to failing the course,
    to expulsion from the university.
  \end{enumerate}
  In other words,
  once you decide to search for answers for a homework assignment,
  if you want to be honest with yourself and others,
  you really have no good options.

  You might try to skirt this fact by telling yourself:
  This course is \emph{really} hard.
  The answers are out there \emph{somewhere} on the internet anyway.
  Nobody is really harmed by this ``one'' infraction.
  I can pretty up someone else's code and pass it off as my own.
  I can put one over on my prof or TA.
  And so forth.

  What you ought to do, however, is ask yourself:
  Do I want to be the kind of person who cheats?
  Am I willing to sell my professional birthright for a quick workaround?
  If I fail to act morally in this little thing,
  how can I expect to be entrusted with big things down the road?
  How will I feel when I'm fired from my job or sued for professional malpractice?

  I know plagiarism is happening.
  I sympathize with the challenges posed by our curriculum.
  I was a student for \emph{way} longer than you
  and faced similar pressure.
  What I'd ask are two things:
  \begin{enumerate}
  \item
    If you have committed plagiarism,
    admit to it,
    take a zero on that work,
    and clear your conscience.
  \item
    Go and sin no more.
    Stop cheating.
    Do your own work.
    Be proud of what you yourself can accomplish
    with the intellectual gifts God gave you.
  \end{enumerate}
\end{quotation}


%%% Local Variables:
%%% mode: latex
%%% End:

% LocalWords:  profilers else's
