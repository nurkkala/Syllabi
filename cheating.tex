As a student at an institution whose goal is to honor Christ in all that it does,
I expect you to uphold the strictest standards of academic integrity.
You must do your own work,
cite others when you present their work,
and never misrepresent your academic performance in any way.
Violation of these standards stains the reputations of you as a student,
Taylor as an institution,
and Jesus as our Lord.

Every assignment should indicate clearly
that it is either:
\begin{itemize}
\item An \textbf{individual} assignment,
  to be done \emph{entirely by you},
  without any direct participation from other students.
\item A \textbf{group} assignment, to be done \emph{collectively with a group}
\end{itemize}
Unless otherwise stated,
assignments are \textbf{individual} assignments.

\begin{flushleft}
  \begin{framed}
    You are \emph{always} welcome
    to get help from the instructor on \emph{any}
    homework assignment or project,
    whether an individual or group assignment.
  \end{framed}
\end{flushleft}

\subsection{What Constitutes Academic Dishonesty?}

For purposes of this course, the following are \emph{non-exhaustive} examples
of violations of academic integrity.
\begin{enumerate}
\item
  Sharing code or other electronic files by copying, retyping, looking at,
  or supplying a copy of a file from this or a previous semester. 
\item
  Sharing written assignments or exams by looking at, copying, or supplying
  an assignment or exam.
\item
  Using another student's code. Using code from this or previous offerings of the
  class, from courses at other institutions, or from any other source (e.g.,
  software found on the Internet).
\item Looking at another student's code. Although mentioned above, it bears
  repeating: looking at other students' code or allowing others to look at yours
  is academic dishonesty. There is no notion of looking ``too much,'' since no looking is
  allowed at all.
\end{enumerate}

\subsection{What Does Not Constitute Academic Dishonesty?}

In contrast, the following are \emph{non-exhaustive} examples of activities
that \emph{do not} violate academic integrity.

\begin{enumerate}
\item Clarifying ambiguities or vague points in class handouts or textbooks.
\item Helping others use the computer systems, networks, compilers, debuggers,
  profilers, or other system facilities without regard to a particular assignment or project.
\item Helping others with high-level design issues.
\item Helping others with high-level (\emph{not} code-based) debugging.
\item Using code provided by the instructor from the course web site or elsewhere.
\end{enumerate}

\subsection{From the Provost}

Taylor's Provost\footnote{At Taylor, the \emph{Provost} is our Chief Academic Officer.}
defines \emph{plagiarism} as follows:
\begin{quote}
  In an instructional setting,
  plagiarism occurs when a person presents or turns in work
  that includes someone else's ideas, language,
  or other (not common-knowledge\footnote{\emph{Common knowledge}
    means any knowledge or facts that could be found in multiple places
    or as defined by a discipline, department, or faculty member.}) material
  without giving appropriate credit to the source.
  Plagiarism will not be tolerated
  and may result in failing this course,
  and may also result in further consequences.
\end{quote}

The Provost goes on to say:
\begin{quote}
  Academic dishonesty constitutes a serious violation
  of academic integrity and scholarship standards at Taylor
  that can result in substantial penalties,
  at the sole discretion of the University,
  including but not limited to,
  denial of credit in a course as well as dismissal from the University.

  In short, a student violates academic integrity
  when he or she claims credit for any work not his or her own
  (words, ideas, answers, data, program codes, music, etc.)
  or when a student misrepresents any academic performance.
\end{quote}

For more information, see the
\href{http://www.taylor.edu/academics/files/undergrad-catalog/current/catalog.pdf}%
{Taylor University Undergraduate Catalog}.

%%% Local Variables:
%%% mode: latex
%%% End:

%  LocalWords:  profilers else's
