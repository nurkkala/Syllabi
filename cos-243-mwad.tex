\documentclass[11pt]{article}

\usepackage{framed}
\usepackage{xcolor}

\newcommand{\syllabuslayout}[3]{
  \usepackage[margin=1.5in]{geometry}
  \usepackage{lastpage}

  \usepackage{fancyhdr}
  \renewcommand{\headrulewidth}{0.4pt}
  \renewcommand{\footrulewidth}{0.4pt}
  \pagestyle{fancy}

  \lhead{Course Syllabus}
  \chead{}
  \rhead{Page \thepage\ of \pageref{LastPage}}
  \lfoot{#1}
  \cfoot{#2}
  \rfoot{#3~\the\year}
}

\newcommand{\titleblock}[2]{
  \begin{center}
    \color{purple}
    {\bfseries
      {\Large #1}\\[4pt]
      {\Large #2}}\\[12pt]
    Department of Computer Science and Engineering\\
    Taylor University
  \end{center}
}

\newcommand{\logistics}[4]{
  \begin{center}
    #1 Credit Hours\\
    #2 $\bullet$\ #3\\[1em]
    Final Exam: #4
  \end{center}
}

\newcommand{\lastupdated}{
\vfill
\begin{flushright}
  \footnotesize
  \textit{This document last updated \today}.
\end{flushright}}

\newenvironment{catalogentry}[1]
{\section{Description}
  \textbf{Prerequisites}: #1\\

  \noindent
  \textbf{From the catalog}:}
{}

\newcommand{\safari}{O'Reilly Learning}
\syllabuslayout{COS~243}{Multi-tier Web Application Development}{Spring}

\usepackage[colorlinks=true,allcolors=blue]{hyperref}
\usepackage{booktabs}
\usepackage{comment}

\begin{document}

\titleblock{COS 243}{Multi-tier Web Application Development}

\logistics{Three}{Euler Science 218}{MWF 9:00--9:50}{W/12-May, 10:00--12:00}

\section{Instructor}

\begin{center}
  \begin{tabular}{rll}
    \toprule
    \multicolumn{3}{l}{\textbf{Dr.\ Tom Nurkkala}}                            \\
    \multicolumn{3}{l}{Associate Professor, Computer Science and Engineering} \\
    \multicolumn{3}{l}{Director, Center for Missions Computing}               \\
    \midrule
    Office & \multicolumn{2}{l}{Euler Science Complex 211}                    \\
    Email  & \multicolumn{2}{l}{\texttt{tnurkkala@cse.taylor.edu}}            \\
    Phone  & \multicolumn{2}{l}{765/998-5163}                                 \\
    Hours  & Mon, Wed & 2:00--3:00                                            \\
           & Tue, Thu & 10:00--12:00                                          \\
           & \multicolumn{2}{l}{\emph{Or by appointment}}                     \\
    \bottomrule
  \end{tabular}
\end{center}

%%% Local Variables:
%%% mode: latex
%%% End:

%  LocalWords:  rrl


\begin{catalogentry}{COS~121, COS~143}
  The course will explore how to develop a complete web application
  with implementation separating concerns between
  content delivery, business logic, and data storage.
  An emphasis on a modern MVC platform will be used
  to provide the separation of concerns.
  Additionally, core database knowledge for a functioning application
  will be explored including data modeling for a relational database,
  common SQL queries, data normalization, foreign key
  constraints, and aggregate operations.
\end{catalogentry}

\section{Learning Objectives}

Upon successful completion of this course,
you will be able to:
\begin{enumerate}
\item Leverage modern web application architecture
\item Store application data in a relational database management system
\item Create entity-relationship diagrams of a data model
\item Use the Git distributed revision control system
  and the \href{https://github.com/}{GitHub} hosting service for Git
\item Employ the massive ecosystem of server-side JavaScript
\item Design and program using asynchronous JavaScript,
  including callbacks and promises
\item Know how to use a database-independent query builder
  to create queries that work across database platforms
\item Understand object-relational mapping
  and apply it in a JavaScript-based web server
\item Leverage the full capability of the HTTP protocol
\item Design and document flexible and powerful RESTful web services
\item Construct a RESTful web server
\item Create tests that validate the behavior of a RESTful server
\item Validate RESTful requests received by a server
\item Design and build a user interface using a standards-compliant toolkit 
\item Implement unit and regression tests for sever- and client-side code
\end{enumerate}

\section{Course Outline}

We will cover the following topics.
\begin{enumerate}
\item Introduction
  \begin{enumerate}
  \item Course Overview
  \item Modern Web Architecture
    \begin{enumerate}
    \item Design Patterns
    \item Example Application
    \end{enumerate}
  \end{enumerate}
\item Data Persistence
  \begin{enumerate}
  \item Relational Database Management
  \item Entity-Relationship Diagrams
  \item Git and GitHub
  \item Server-Side JavaScript
  \item Asynchronous JavaScript
  \item Query Building
  \item Object-Relational Mapping
  \end{enumerate}
\item RESTful Web Services
  \begin{enumerate}
  \item HTTP
  \item RESTful APIs
    \begin{enumerate}
    \item Design Guidelines
    \item Endpoints
    \item Testing
    \end{enumerate}
  \item RESTful API Server
    \begin{enumerate}
    \item Concepts
    \item Server
    \item Validation
    \item Testing
    \end{enumerate}
  \end{enumerate}
\item Single-Page Applications
  \begin{enumerate}
  \item UI Toolkits
  \item Development Tools
  \item Authentication
  \item Testing
  \end{enumerate}
\end{enumerate}

\section{Project}

During the course,
you will build a modern, full-stack web application
using all the technologies we will cover.

\section{Text}

Rather than selecting a traditional textbook
that you are ``forced'' to purchase by the bookstore,
I encourage you to take advantage of 
\href{https://www.safaribooksonline.com/}{Safari Books Online}.
Safari is an outstanding subscription service that gives you access
to thousands of technical titles,
including many for this course.
You can also download complete books for off-line
access on your mobile devices.

For students, the cost to subscribe directly is about \$40/month.
Access is also included in the very affordable student membership in the
\href{https://www.acm.org/}{Association for Computing Machinery} (ACM),
with the \href{https://www.computer.org/}{IEEE Computer Society},
one of the two main professional societies for computer scientists.

Why use Safari instead of a standard textbook?
\begin{enumerate}
\item
  Rather than asking you to buy an entire textbook
  and use only \emph{portions} of it,
  we will be draw on multiple resources
  that are \emph{directly related} to the topics 
  covered throughout the semester.
\item
  Your total cost will be about \$120,
  which is less than the cost of some single textbooks.
  If you are taking other courses from me this semester,
  you only need to pay the fee once.
\item
  You should have experience learning from written
  and electronic resources, which you will do
  throughout your technical career.
  I have been a member of Safari for many years
  and find it to be an invaluable resource
  when I am learning a new technology
  or buffing up my understanding of a familiar one.
\item
  You will have immediate access to a rich collection
  of technical material to advance your learning in this
  or other computer science courses.
\end{enumerate}

%%% Local Variables:
%%% mode: latex
%%% End:

% LocalWords:  ACM


\section{Evaluation}

The grading breakdown for the course
is shown in Table~\ref{tab:grading}.
Refer to my \emph{Periodic Table of the Grades}
for the grading scheme.
I reserve the right to award a higher grade than strictly earned;
outstanding attendance and class participation
figure prominently in such decisions.

\begin{table}[htb]
  \centering
  % BEGIN RECEIVE ORGTBL grades
  \begin{tabular}{llr}
    \toprule
    Category &                 & Weight \\
    \midrule
    Homework & Reading Quizzes & 10\%   \\
             & Programming     & 30\%   \\
    Project  &                 & 25\%   \\
    Exams    & Midterm         & 15\%   \\
             & Final           & 20\%   \\
    \midrule
             & Total           & 100\%  \\
    \bottomrule
  \end{tabular}
  % END RECEIVE ORGTBL grades
  \caption{Grading details}
  \label{tab:grading}
\end{table}
\begin{comment}
#+ORGTBL: SEND grades orgtbl-to-latex :splice nil :skip 0 :booktabs t
| Category |                 | Weight |
|          |                 |    <r> |
|----------+-----------------+--------|
| Homework | Reading Quizzes |    10% |
|          | Programming     |    30% |
| Project  |                 |    25% |
| Exams    | Midterm         |    15% |
|          | Final           |    20% |
|----------+-----------------+--------|
|          | Total           |   100% |
#+TBLFM: @9$3=100*vsum(@I..II);%d%%
\end{comment}

\section{Course Expectations}

Following are my expectations regarding the course.

\subsection{Attendance}
\label{sec:attendance}

You are required to attend all class sessions.
I will be in class each day, and I expect you to be there also.

In general, I am very understanding about students who must miss class
due to a sanctioned Taylor activity, medical appointment, job interview,
family emergency, and the like.
If possible, let me know in advance that you will not be in class;
I will work with you to arrange make-up instruction, homework, exams, etc.

\subsection{Late Work}

All course assignments will include an unambiguous due date.
Usually, assignments are due at the beginning of class on the due date.
If there are multiple sections of a class,
the assignment is due at the beginning of the earliest such section.
Barring exceptional circumstances like those mentioned in section~\ref{sec:attendance},
I expect your work to be submitted \emph{on the due date}.
Late work will \emph{not} be accepted.

This policy on late work is intended to prepare you
for real-world experience after graduation.
In the marketplace,
late work is not merely an inconvenience.
Missing a deadline may
alienate your customer,
upset your manager,
ruin your project,
or terminate your employment!
\emph{Now} is the time to learn the self discipline and time management skills
required to complete your work when it is due.

\subsection{Conduct}

I expect you to be prepared, awake, aware, and participatory during class. I will not
hesitate to ask you to stand or move if you are distracted or sleepy.

I expect you to join in discussions, respond to questions from me and from your
colleagues, and ask questions of me. I expect you to hold my feet to the fire if I am
being unclear, unkind, or contradictory.

\subsection{Gizmos}

You may not use a laptop, tablet, or similar device to check e-mail, engage in social
networking, surf the web, or any other activity not directly relevant
to current classroom activity.
If you use an electronic gizmo during class for legitimate academic purposes
(e.g., note taking), be prepared to demonstrate relevant use on demand
at any time.

\section{Moodle}

The Computer Science and Engineering department uses Moodle
as our Learning Management System.
The URL for Moodle is \url{https://moodle.cse.taylor.edu}.
To sign on to the course site for the first time,
you will need an enrollment key.
The key for this course is
\texttt{nerds4christ}.

You are responsible for checking Moodle regularly
to keep up with assignment due dates and other announcements.
For due dates, \emph{the Moodle calendar is your friend}.

\section{Slack}

This course will use Slack
for informal communication,
Q\&A,
last minute announcements,
jokes,
and the like.
You are \emph{strongly} encouraged to join the conversation.

Find the \emph{TU CSE Student} slack team at
\url{tucsestudents.slack.com}.
Look there for a \emph{channel}
dedicated to the course.

\section{Academic Integrity}

As a student at an institution whose goal is to honor Christ in all that it does,
I expect you to uphold the strictest standards of academic integrity.
You must do your own work,
cite others when you present their work,
and never misrepresent your academic performance in any way.
Violation of these standards stains the reputations of you as a student,
Taylor as an institution,
and Jesus as our Lord.

Every assignment should indicate clearly
that it is either:
\begin{itemize}
\item An \textbf{individual} assignment,
  to be done \emph{entirely by you},
  without any direct participation from other students.
\item A \textbf{group} assignment, to be done \emph{collectively with a group}
\end{itemize}
Unless otherwise stated,
assignments are \textbf{individual} assignments.

\begin{framed}
Note that you are \emph{always} welcome
to get help from the instructor.
\end{framed}

A violation of academic integrity may result in your failing the course
and other disciplinary action by the University.
Refer to the Taylor catalog for the official statement of these ideas.

%  LocalWords:  christ

%%% Local Variables:
%%% mode: latex
%%% TeX-master: "sys394-isd"
%%% End:


\section{Pandemic}

\section{Pandemic}

We will adhere to university guidance regarding the COVID-19 pandemic.
\begin{quote}
  % For AY 2021-2022:
  Taylor University is committed to its academic and spiritual mission,
  even as we teach and learn in a global pandemic.
  \begin{enumerate}
  \item As students, you can expect
    that your professor will communicate clearly and regularly
    about expectations for this course.
  \item 
    Should you need to be absent from class at any time,
    it is your responsibility to notify your professors.
  \item 
    Should it be determined that you will be absent due to isolation or quarantine,
    it is your responsibility to notify your professors.
    Professors will help you stay current with course content
    using tools appropriate to that content.
    Communication in a timely fashion will enable this to be possible.
  \item 
    If course delivery methods need to change or adapt to health concerns,
    updates will be announced in class sessions.
  \end{enumerate}
  
  % For AY 2020-2021:
  % Taylor University is committed to its academic and spiritual mission,
  % even as we teach and learn in a global pandemic.
  % In the interest of your safety,
  % and the health and wellness of others in the classroom, campus, and community,
  % you can expect to have social distancing and precautionary measures used in the classroom.
  % \textbf{This means you are required to wear your own face mask in class}.
  % You may also be asked to remain seated at a distance
  % away from other students when you are in the classroom.
  % Some larger classes may create discussion groups
  % and establish a rotation for some students to join the class
  % using virtual tools.
  % As students,
  % you can expect that your professor will communicate clearly and regularly
  % about expectations for this course.
  % If course delivery methods need to change or adapt to health concerns,
  % updates will be announced in class sessions.
\end{quote}

%%% Local Variables:
%%% mode: latex
%%% End:


\section{Course Management}

We use several systems to help manage the course
and for on-line communication.

\subsection{Email}

Electronic mail is an official channel of communication
between all members of the university community.
You are responsible to check your email regularly
(daily)
for information related to the course.


\subsection{Canvas}

The Computer Science and Engineering department uses Canvas
as our Learning Management System.
The URL for Canvas is \url{https://canvas.cse.taylor.edu}.

You are responsible for checking Canvas regularly
to keep up with assignment due dates and other announcements.
For due dates, \emph{the Canvas calendar is your friend}.

\subsection{Slack}

This course will use Slack
for informal communication,
Q\&A,
last minute announcements,
jokes,
and the like.
Find the \emph{TU CSE Student} slack team at
\url{tucsestudents.slack.com}.
Look there for a \emph{channel}
dedicated to the course.


\section{Final Exam Policy}

\section{Final Exam Policy}

Students must take their final examinations at the assigned hours
listed in the schedule of classes.
Exceptions will be made only because of serious illness
or the death of an immediate member of the family.
Reasons such as
plane schedules,
availability of flights,
and rides leaving early
are not acceptable exceptions.
Students having three or more examinations on the same
should report this to the Registrar's Office
\emph{ten days prior to the beginning of finals week}.
Reasonable alternatives in alleviating this dilemma
will be pursued by the registrar
and the student in consultation with the appropriate faculty.


\section{Academic Integrity}

\section{Academic Integrity}

As a student at an institution whose goal is to honor Christ in all that it does,
I expect you to uphold the strictest standards of academic integrity.
You must do your own work,
cite others when you present their work,
and never misrepresent your academic performance in any way.
Violation of these standards stains the reputations of you as a student,
Taylor as an institution,
and Jesus as our Lord.

Every assignment should indicate clearly
that it is either:
\begin{itemize}
\item An \textbf{individual} assignment,
  to be done \emph{entirely by you},
  without any direct participation from other students.
\item A \textbf{group} assignment, to be done \emph{collectively with a group}
\end{itemize}
Unless otherwise stated,
assignments are \textbf{individual} assignments.

\begin{flushleft}
  \begin{framed}
    You are \emph{always} welcome
    to get help from the instructor on \emph{any}
    homework assignment or project,
    whether an individual or group assignment.
  \end{framed}
\end{flushleft}

\subsection{What Constitutes Academic Dishonesty?}
\label{sec:what-is}

For purposes of this course, the following are \emph{non-exhaustive} examples
of violations of academic integrity.
\begin{enumerate}
\item
  Sharing code or other electronic files by copying, retyping, looking at,
  or supplying a copy of a file from this or a previous semester. 
\item
  Sharing written assignments or exams by looking at, copying, or supplying
  an assignment or exam.
\item
  Using another student's code. Using code from this or previous offerings of the
  class, from courses at other institutions, or from any other source (e.g.,
  software found on the Internet).
\item\label{i:looking}
  Looking at another student's code. Although mentioned above, it bears
  repeating: looking at other students' code or allowing others to look at yours
  is academic dishonesty. There is no notion of looking ``too much,'' since no looking is
  allowed at all.
\end{enumerate}

\subsection{What Does Not Constitute Academic Dishonesty?}

In contrast, the following are \emph{non-exhaustive} examples of activities
that \emph{do not} violate academic integrity.

\begin{enumerate}
\item Clarifying ambiguities or vague points in class handouts or textbooks.
\item Helping others use the computer systems, networks, compilers, debuggers,
  profilers, or other system facilities without regard to a particular assignment or project.
\item Helping others with high-level design issues.
\item Helping others with high-level (\emph{not} code-based) debugging.
\item Using code provided by the instructor from the course web site or elsewhere.
\end{enumerate}

\subsection{From the Provost}

Taylor's Provost\footnote{At Taylor, the \emph{Provost} is our Chief Academic Officer.}
defines \emph{plagiarism} as follows:
\begin{quote}
  In an instructional setting,
  plagiarism occurs when a person presents or turns in work
  that includes someone else's ideas, language,
  or other (not common-knowledge\footnote{\emph{Common knowledge}
    means any knowledge or facts that could be found in multiple places
    or as defined by a discipline, department, or faculty member.}) material
  without giving appropriate credit to the source.
  Plagiarism will not be tolerated
  and may result in failing this course,
  and may also result in further consequences
  as stipulated in the
  \href{http://www.taylor.edu/academics/files/undergrad-catalog/current/catalog.pdf}{Taylor catalog}.
\end{quote}

The Provost goes on to say:
\begin{quote}
  Academic dishonesty constitutes a serious violation
  of academic integrity and scholarship standards at Taylor
  that can result in substantial penalties,
  at the sole discretion of the University,
  including but not limited to,
  denial of credit in a course as well as dismissal from the University.

  In short, a student violates academic integrity
  when he or she claims credit for any work not his or her own
  (words, ideas, answers, data, program codes, music, etc.)
  or when a student misrepresents any academic performance.
  Please see 
  \href{http://www.taylor.edu/academics/files/undergrad-catalog/current/catalog.pdf}%
  {the catalog} for a complete statement.
\end{quote}

\subsection{Personal Reflection}

I wrote an initial version of this section
as a Slack message
in the wake of several students who admitted to cribbing code
for homework assignments
from the Internet:
\begin{quotation}
  \textbf{Plagiarism}---We all look to the Internet
  to search for quick solutions to all manner of computing problems:
  checking a function signature,
  getting more detail on the meaning of an error code,
  recalling the behavior of a language construct,
  or finding a canonical implementation of a standard algorithm.

  Professional honesty and integrity demand
  that if we employ the results of such a search in our own work,
  we do so only if:
  \begin{enumerate}
  \item
    It is permitted (e.g., if the material is posted publicly)
  \item 
    We cite the source (e.g., a comment in our code).
  \end{enumerate}
  This is true whether we're working on a project in college
  or on a multi-year product after we've been in professional practice for decades.
  
  Searching for documentation on a function,
  however,
  is \emph{qualitatively different}
  from searching for the solution to a homework assignment.
  The former search is a quick and effective way
  to shore up our understanding.
  The latter is the gateway to academic dishonesty: cheating by plagiarism.
  
  As mentioned (Section~\ref{sec:what-is}, item \ref{i:looking}),
  even \emph{looking} at code written by someone else
  as you try to solve a class assignment is expressly forbidden.
  The reason for this prohibition is simple:
  if you find someone else's solution to a homework assignment,
  you are you longer even \emph{able} to solve the problem independently
  (or with your team).
  You can't ``un-see'' the existing solution.
  Your solution can no longer be your own---it's your repackaging of someone else's work.
  You are now faced with three options:
  \begin{enumerate}
  \item
    Don't submit a solution at all
    because any solution you construct
    would not be your own work.
    You will receive no credit.
  \item
    Submit the work with an explicit citation
    of the source from which it is derived.
    You avoid the charge of plagiarism,
    but you will also receive no credit for the work
    because it's not yours.  
  \item
    Submit the work as if it is your own
    without citing the source.
    This is \emph{outright plagiarism}.
    According to university and course policy,
    it could result in penalties ranging from
    a zero on the assignment,
    to failing the course,
    to expulsion from the university.
  \end{enumerate}
  In other words,
  once you decide to search for answers for a homework assignment,
  if you want to be honest with yourself and others,
  you really have no good options.

  You might try to skirt this fact by telling yourself:
  This course is \emph{really} hard.
  The answers are out there \emph{somewhere} on the internet anyway.
  Nobody is really harmed by this ``one'' infraction.
  I can pretty up someone else's code and pass it off as my own.
  I can put one over on my prof or TA.
  And so forth.

  What you ought to do, however, is ask yourself:
  Do I want to be the kind of person who cheats?
  Am I willing to sell my professional birthright for a quick workaround?
  If I fail to act morally in this little thing,
  how can I expect to be entrusted with big things down the road?
  How will I feel when I'm fired from my job or sued for professional malpractice?

  I know plagiarism is happening.
  I sympathize with the challenges posed by our curriculum.
  I was a student for \emph{way} longer than you
  and faced similar pressure.
  What I'd ask are two things:
  \begin{enumerate}
  \item
    If you have committed plagiarism,
    admit to it,
    take a zero on that work,
    and clear your conscience.
  \item
    Go and sin no more.
    Stop cheating.
    Do your own work.
    Be proud of what you yourself can accomplish
    with the intellectual gifts God gave you.
  \end{enumerate}
\end{quotation}


%%% Local Variables:
%%% mode: latex
%%% End:

% LocalWords:  profilers else's


\section{Support Services}

Be aware of the following support services
available to you as a Taylor student.

\subsection{Academic Assistance}

The Academic Enrichment Center (AEC), located in the Zondervan Library,
provides individualized academic skills help
(e.g. test preparation, note taking, planning, etc.).
Contact \textbf{Dr.\ Scott Gaier}, \texttt{scgaier@taylor.edu}.
 
\subsection{Tutoring}

Peer Tutoring Services,
located in the AEC in Zondervan Library,
provides free help to students in most content areas.
For further information, contact
\textbf{Darci Nurkkala}, \texttt{drnurkkala@taylor.edu}.

\subsection{Students with Special Needs}

The Academic Enrichment Center provides a variety of services
for students who have disabilities.
This includes, but is not limited to, mental, emotional, physical, and learning disabilities.
Contact \textbf{Dr.\ Scott Barrett}, \texttt{scott\_barrett@taylor.edu}, to learn more.
If you need accommodations due to a disability,
please also see me so that I can help accordingly.

\subsection{Writing Center}

Writing Center tutors can help you on all of your writing
in any stage of your writing process,
but they will usually focus on content and organization
before they look at grammar and style.
Expect to be actively involved during your session,
whether you are developing a better thesis,
reorganizing your main points,
or consulting a style manual to understand formatting rules.
To arrange an appointment visit \url{taylor.mywconline.com}. 

% LocalWords:  Zondervan Gaier AEC


\lastupdated

\end{document}

%%% Local Variables:
%%% mode: latex
%%% TeX-master: t
%%% End:

% LocalWords:  llr MWF ORGTBL orgtbl booktabs TBLFM vsum
