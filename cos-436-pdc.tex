\documentclass[11pt]{article}

\usepackage{framed}
\usepackage{xcolor}

\newcommand{\syllabuslayout}[3]{
  \usepackage[margin=1.5in]{geometry}
  \usepackage{lastpage}

  \usepackage{fancyhdr}
  \renewcommand{\headrulewidth}{0.4pt}
  \renewcommand{\footrulewidth}{0.4pt}
  \pagestyle{fancy}

  \lhead{Course Syllabus}
  \chead{}
  \rhead{Page \thepage\ of \pageref{LastPage}}
  \lfoot{#1}
  \cfoot{#2}
  \rfoot{#3~\the\year}
}

\newcommand{\titleblock}[2]{
  \begin{center}
    \color{purple}
    {\bfseries
      {\Large #1}\\[4pt]
      {\Large #2}}\\[12pt]
    Department of Computer Science and Engineering\\
    Taylor University
  \end{center}
}

\newcommand{\logistics}[4]{
  \begin{center}
    #1 Credit Hours\\
    #2 $\bullet$\ #3\\[1em]
    Final Exam: #4
  \end{center}
}

\newcommand{\lastupdated}{
\vfill
\begin{flushright}
  \footnotesize
  \textit{This document last updated \today}.
\end{flushright}}

\newenvironment{catalogentry}[1]
{\section{Description}
  \textbf{Prerequisites}: #1\\

  \noindent
  \textbf{From the catalog}:}
{}

\newcommand{\safari}{O'Reilly Learning}
\syllabuslayout{COS~436}{Parallel and Distributed Computing}{Spring}

\usepackage[colorlinks=true,allcolors=blue]{hyperref}
\usepackage{booktabs}
\usepackage{comment}

\usepackage{biblatex}
\addbibresource{courses.bib}

\begin{document}

\section{Instructor}

\begin{center}
  \begin{tabular}{rll}
    \toprule
    \multicolumn{3}{l}{\textbf{Dr.\ Tom Nurkkala}}                            \\
    \multicolumn{3}{l}{Associate Professor, Computer Science and Engineering} \\
    \multicolumn{3}{l}{Director, Center for Missions Computing}               \\
    \midrule
    Office & \multicolumn{2}{l}{Euler Science Complex 211}                    \\
    Email  & \multicolumn{2}{l}{\texttt{tnurkkala@cse.taylor.edu}}            \\
    Phone  & \multicolumn{2}{l}{765/998-5163}                                 \\
    Hours  & Mon, Wed & 2:00--3:00                                            \\
           & Tue, Thu & 10:00--12:00                                          \\
           & \multicolumn{2}{l}{\emph{Or by appointment}}                     \\
    \bottomrule
  \end{tabular}
\end{center}

%%% Local Variables:
%%% mode: latex
%%% End:

%  LocalWords:  rrl


\section{Course Overview}

Most---if not all---of the courses
you have taken so far in computer science
have focused on algorithms and programs
for a single processor.
This approach to learning makes a great deal of sense.
It's plenty hard to write and debug programs
on a single processor;
why complicate matters with more than one?

For a variety of reasons that we'll discuss,
the world hasn't run exclusively on one-processor computers
for a long time.
Back in~2001, IBM released the first
two-core microprocessor---the Power4.\cite{ibm-power4}
As we'll see,
since that watershed introduction,
multi-core, many-core, and massively parallel processors
have taken the world by storm.
You must understand these architectures
and be able to write and debug programs on them
in order to be competitive in the modern computing landscape.

In this course,
we will study
multi-core CPUs,
many-core GPUs,
and a variety of tools, technologies, and techniques
that will help you
design and write software
that maximizes the performance of modern computer systems.

\section{Learning Objectives}

Upon successful completion of this course, you should be able to:



\begin{enumerate}
\item Explain the history of high-performance computing
  and place modern parallel and distributed computing
  in its historical context.
\item Differentiate parallel computing
  from distributed computing.
\item Differentiate parallelism and concurrency.
\item Explain Flynn's taxonomy of parallel architectures.
\item Differentiate uniprocessor, many-core, and multi-core computer architectures.
\item Apply decomposition and design strategies
  for various parallel problems and architectures.
\item Formulate and implement efficient parallel versions of sequential algorithms.
\item Measure parallel algorithm performance.
\item Calculate parallel speedup and efficiency.
\item Employ a threaded model of parallel computation,
  including common parallel programming constructs
  like semaphores, shared memory, and monitors.
\item Build correct and performant software
  for both shared- and distributed-memory multicomputers
  using OpenMP and MPI.
\item Understand the host-device model
  for modern, general-purpose graphics processing units (GPUs).
\item Build correct and performant software for GPUs.
\item Construct hybrid parallel programs
  that execute cooperatively on both CPU and GPU cores.
\end{enumerate}

\section{Texts}

There is one \textbf{required} text:
\begin{itemize}
\item Barlas~\cite{barlas2014multicore}
\end{itemize}
You may also wish to acquire access to the following \textbf{recommended} texts:
\begin{itemize}
\item Cheng, et.\ al.~\cite{cheng2014professional}
\item Wilt~\cite{wilt2013cuda}
\item Storti and Yurtoglu~\cite{storti2015cuda}
\end{itemize}
You can order the textbook online,
but you may also wish to consider joining Safari Books Online~\cite{safari}.
Over the course of the semester,
Safari will cost about the same as the price of Barlas,
but will give you access to the other resources mentioned above
\emph{and} a ton of other great technical books, videos, and other resources.
I have been a member of Safari for many years,
and I find it to be an invaluable technical resource.\footnote{I receive no compensation
  from Safari for this endorsement.}

\section{Evaluation}

The grading breakdown
for the course
is shown in table~\ref{tab:grading}.
Refer to my \emph{Periodic Table of the Grades} (on Moodle) for the grading scheme. I reserve
the right to award a higher grade than strictly earned; outstanding attendance and class
participation figure prominently in such decisions.

\begin{table}[htb]
  \centering
  % BEGIN RECEIVE ORGTBL grades
  \begin{tabular}{lr}
    \toprule
    Category & Weight \\
    \midrule
    Homework & 35\%   \\
    Exam 1   & 20\%   \\
    Exam 2   & 20\%   \\
    Final    & 25\%   \\
    \midrule
    Total    & 100\%  \\
    \bottomrule
  \end{tabular}
  % END RECEIVE ORGTBL grades
  \caption{Grading details}
  \label{tab:grading}
\end{table}

\begin{comment}
  #+ORGTBL: SEND grades orgtbl-to-latex :booktabs t
  | Category | Weight |
  |----------+--------|
  | <l>      |    <r> |
  | Homework |    35% |
  | Exam 1   |    20% |
  | Exam 2   |    20% |
  | Final    |    25% |
  |----------+--------|
  | Total    |   100% |
\end{comment}

\section{Course Expectations}

Following are my expectations regarding the course.

\subsection{Attendance}
\label{sec:attendance}

You are required to attend all class sessions.
I will be in class each day, and I expect you to be there also.

In general, I am very understanding about students who must miss class
due to a sanctioned Taylor activity, medical appointment, job interview,
family emergency, and the like.
If possible, let me know in advance that you will not be in class;
I will work with you to arrange make-up instruction, homework, exams, etc.

\subsection{Late Work}

All course assignments will include an unambiguous due date.
Usually, assignments are due at the beginning of class on the due date.
If there are multiple sections of a class,
the assignment is due at the beginning of the earliest such section.
Barring exceptional circumstances like those mentioned in section~\ref{sec:attendance},
I expect your work to be submitted \emph{on the due date}.
Late work will \emph{not} be accepted.

This policy on late work is intended to prepare you
for real-world experience after graduation.
In the marketplace,
late work is not merely an inconvenience.
Missing a deadline may
alienate your customer,
upset your manager,
ruin your project,
or terminate your employment!
\emph{Now} is the time to learn the self discipline and time management skills
required to complete your work when it is due.

\subsection{Conduct}

I expect you to be prepared, awake, aware, and participatory during class. I will not
hesitate to ask you to stand or move if you are distracted or sleepy.

I expect you to join in discussions, respond to questions from me and from your
colleagues, and ask questions of me. I expect you to hold my feet to the fire if I am
being unclear, unkind, or contradictory.

\subsection{Gizmos}

You may not use a laptop, tablet, or similar device to check e-mail, engage in social
networking, surf the web, or any other activity not directly relevant
to current classroom activity.
If you use an electronic gizmo during class for legitimate academic purposes
(e.g., note taking), be prepared to demonstrate relevant use on demand
at any time.

\section{Moodle}

The Computer Science and Engineering department uses Moodle
as our Learning Management System.
The URL for Moodle is \url{https://moodle.cse.taylor.edu}.
To sign on to the course site for the first time,
you will need an enrollment key.
The key for this course is
\texttt{nerds4christ}.

You are responsible for checking Moodle regularly
to keep up with assignment due dates and other announcements.
For due dates, \emph{the Moodle calendar is your friend}.

\section{Slack}

This course will use Slack
for informal communication,
Q\&A,
last minute announcements,
jokes,
and the like.
You are \emph{strongly} encouraged to join the conversation.

Find the \emph{TU CSE Student} slack team at
\url{tucsestudents.slack.com}.
Look there for a \emph{channel}
dedicated to the course.

\section{Academic Integrity}

As a student at an institution whose goal is to honor Christ in all that it does,
I expect you to uphold the strictest standards of academic integrity.
You must do your own work,
cite others when you present their work,
and never misrepresent your academic performance in any way.
Violation of these standards stains the reputations of you as a student,
Taylor as an institution,
and Jesus as our Lord.

Every assignment should indicate clearly
that it is either:
\begin{itemize}
\item An \textbf{individual} assignment,
  to be done \emph{entirely by you},
  without any direct participation from other students.
\item A \textbf{group} assignment, to be done \emph{collectively with a group}
\end{itemize}
Unless otherwise stated,
assignments are \textbf{individual} assignments.

\begin{framed}
Note that you are \emph{always} welcome
to get help from the instructor.
\end{framed}

A violation of academic integrity may result in your failing the course
and other disciplinary action by the University.
Refer to the Taylor catalog for the official statement of these ideas.

%  LocalWords:  christ

%%% Local Variables:
%%% mode: latex
%%% TeX-master: "sys394-isd"
%%% End:


\printbibliography{}

\end{document}

%%% Local Variables:
%%% mode: latex
%%% TeX-master: t
%%% End:

%  LocalWords:  ISD Kepner Tregoe SRS ISA llllll christ multi OpenMP MPI GPUs
%  LocalWords:  GPU Barlas Cheng et al Storti Yurtoglu
