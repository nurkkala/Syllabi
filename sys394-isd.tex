\documentclass[11pt]{article}

\usepackage{framed}
\usepackage{xcolor}

\newcommand{\syllabuslayout}[3]{
  \usepackage[margin=1.5in]{geometry}
  \usepackage{lastpage}

  \usepackage{fancyhdr}
  \renewcommand{\headrulewidth}{0.4pt}
  \renewcommand{\footrulewidth}{0.4pt}
  \pagestyle{fancy}

  \lhead{Course Syllabus}
  \chead{}
  \rhead{Page \thepage\ of \pageref{LastPage}}
  \lfoot{#1}
  \cfoot{#2}
  \rfoot{#3~\the\year}
}

\newcommand{\titleblock}[2]{
  \begin{center}
    \color{purple}
    {\bfseries
    {\Large #1}\\[4pt]
    {\Large #2}}\\[12pt]
    Department of Computer Science and Engineering\\
    Taylor University
  \end{center}
}

\syllabuslayout{SYS~394}{Information Systems Design}{Spring}

\usepackage[colorlinks=true,allcolors=blue]{hyperref}
\usepackage{booktabs}
\usepackage{comment}

\usepackage{biblatex}
\addbibresource{courses.bib}

\begin{document}

\section{Instructor}

\begin{center}
  \begin{tabular}{rll}
    \toprule
    \multicolumn{3}{l}{\textbf{Dr.\ Tom Nurkkala}}                            \\
    \multicolumn{3}{l}{Associate Professor, Computer Science and Engineering} \\
    \multicolumn{3}{l}{Director, Center for Missions Computing}               \\
    \midrule
    Office & \multicolumn{2}{l}{Euler Science Complex 211}                    \\
    Email  & \multicolumn{2}{l}{\texttt{tnurkkala@cse.taylor.edu}}            \\
    Phone  & \multicolumn{2}{l}{765/998-5163}                                 \\
    Hours  & Mon, Wed, Thu, Fri & 11:00--12:00                                \\
           & Mon, Wed           & 2:00--3:00                                  \\
           & \multicolumn{2}{l}{\emph{Or by appointment}}                     \\
    \bottomrule
  \end{tabular}
\end{center}

%%% Local Variables:
%%% mode: latex
%%% End:

%  LocalWords:  rrl


\section{Course Overview}

This course focuses on the physical design and construction phases
of the system development life cycle.
Our goal is to implement a working prototype
of the system defined and analyzed in SYS~390~(ISA)
as a database-backed web application.
We will cover web and database programming,
advanced SQL topics,
software project management practices,
system architecture,
and related web technology.

The course comprises two hands-on, lab-heavy segments.
\begin{enumerate}
\item
  For approximately the first five weeks,
  we will focus on the basics of developing
  a database-backed web application.
  This portion of the course is designed
  to provide everyone in the class
  with the knowledge and skills
  necessary to understand and construct a non-trivial web application.
  We will cover salient topics in class,
  and you will complete a series of lab exercises
  that give you practice with the topic of the day.
\item
  For the remaining ten weeks of the course,
  you will and your team will construct a prototype application
  that implements the requirements you gathered in ISA
  as reflected in your System Requirements Specification (SRS).
\end{enumerate}

\section{Learning Objectives}

When you complete this class you should be able to:
\begin{enumerate}
\item Understand the importance
  of a formal system development methodology
  to information technology project success
\item Translate a logical design to a physical design
\item Apply user interface design skills
\item Employ agile software project management practices and tools
\item Participate effectively on a software development team
\item Demonstrate relational database skills
  in both SQL data definition and data manipulation
\item Make effective use of distributed revision control.
\item Use key technologies powering modern web applications:
  three-tier architecture, web pages,
  development tools, and servers
\item Apply testing skills,
  including unit tests and client-side GUI tests
\item Present effectively the results of your work to others
\end{enumerate}

\section{Text}

There is no required text for the course.

\section{Project}

We will continue the project begun in ISA.
Our goal is to build working prototypes
of a system that implements the bulk of the requirements
you envisioned in your SRS in ISA.

\subsection{Teams}

Again this semester, you will work in teams,
typically comprising 3--4 students.
Unlike ISA, however, I will select the members of the teams
in order to ensure each team has a fair balance
of technical and non-technical expertise.

I will distribute a brief survey early in the semester
that will help me gauge your technical background.
The survey also asks you to indicate who you would (and would not) like
to have on your team.
Although a good balance of skills will override your personal preferences,
I do my best to honor your request.

\subsection{Development Process}

Teams will employ a development process
based on the standard agile process known as \emph{Scrum}.
We will employ an iterative approach to project development
in which your team will focus on delivering
small increments of functionality
that provide both you and me a clear and demonstrable indication of your progress.

As in ISA, \textbf{procrastination is the enemy}.
Using an agile, iterative process forces you and your team
regularly to deliver working software,
which not only gives you a clear view of your accomplishments,
but helps stem the tide of procrastination.

When the project is under way,
we will hold a weekly \emph{Critical Design Review} (CDR).
The CDR gives you a chance to present your progress to the class
and to practice in various roles on a software development team.

\subsection{Show-and-Tell}

At the end of the semester,
we will have a ``show and tell'' time,
during which you will demonstrate your prototype application.
Our customer will be invited to join us for your demonstrations,
either in person or electronically.

Although this presentation is not \emph{nearly} as formal
as your final presentation in ISA,
it is nonetheless your opportunity
to show off the fruit of a year of your labor.
Begin the semester with this end in mind,
and stay focused on delivering an excellent and exciting final product!

\subsection{Technology}

For lab assignments and the project,
we will be using the following technologies (among others).
\begin{itemize}
\item Standard tools and libraries in the Python ecosystem,
  allowing you to leverage your experience
  from COS~120 and any other exposure you've had to Python.
\item The \href{http://flask.pocoo.org/}{Flask}
  framework for web development in Python.
\item The \href{https://git-scm.com/}{Git}
  distributed revision control system.
  We will use Git
  for several purposes, including to:
  \begin{itemize}
  \item Distribute and collect lab assignments
  \item Coordinate work among your team members
  \item Provide insight into your work for grading and guidance
  \end{itemize}
\item \href{https://github.com/}{Github},
  a common location for sharing Git resources
\item The (optional, but highly recommended)
  \href{https://www.jetbrains.com/pycharm/}{PyCharm}
  integrated development environment (IDE).
\end{itemize}

\section{Evaluation}

The grading breakdown for the course
is shown in table~\ref{tab:grading}.
Refer to my \emph{Periodic Table of the Grades} (on Moodle)
for the grading scheme.
I reserve the right to award a higher grade than strictly earned;
outstanding attendance and class participation
figure prominently in such decisions.

\begin{table}[htb]
  \centering
  \begin{tabular}{llrr}
    \toprule
    Category &                    &       & Weight \\
    \midrule
    Labs     &                    &       & 40\%   \\
    Project  &                    &       & 60\%   \\
             & Github Activity    & 40\%  &        \\
             & Total Hours        & 40\%  &        \\
             & CDRs               & 10\%  &        \\
             & Final Presentation & 10\%  &        \\
             &                    & 100\% &        \\
    \midrule
    Total    &                    &       & 100\%  \\
    \bottomrule
  \end{tabular}
  \caption{Grading details}
  \label{tab:grading}
\end{table}

\section{Course Expectations}

\section{Course Expectations}

Following are my expectations regarding the course.

\subsection{Attendance}
\label{sec:attendance}

You are required to attend all class sessions.
I will be in class each day, and I expect you to be there also.

In general, I am very understanding about students who must miss class
due to a sanctioned Taylor activity, medical appointment, job interview,
family emergency, and the like.
If possible, let me know in advance that you will not be in class;
I will work with you to arrange make-up instruction, homework, exams, etc.

\subsection{Late Work}

All course assignments will include an unambiguous due date.
Usually, assignments are due at the beginning of class on the due date.
If there are multiple sections of a class,
the assignment is due at the beginning of the earliest such section.
Barring exceptional circumstances like those mentioned in section~\ref{sec:attendance},
I expect your work to be submitted \emph{on the due date}.
Late work will \emph{not} be accepted.

This policy on late work is intended to prepare you
for real-world experience after graduation.
In the marketplace,
late work is not merely an inconvenience.
Missing a deadline may
alienate your customer,
upset your manager,
ruin your project,
or terminate your employment!
\emph{Now} is the time to learn the self discipline and time management skills
required to complete your work when it is due.

\subsection{Conduct}

I expect you to be prepared, awake, aware, and participatory during class. I will not
hesitate to ask you to stand or move if you are distracted or sleepy.

I expect you to join in discussions, respond to questions from me and from your
colleagues, and ask questions of me. I expect you to hold my feet to the fire if I am
being unclear, unkind, or contradictory.

\subsection{Gizmos}

You may not use a laptop, tablet, or similar device to check e-mail, engage in social
networking, surf the web, or any other activity not directly relevant
to current classroom activity.
If you use an electronic gizmo during class for legitimate academic purposes
(e.g., note taking), be prepared to demonstrate relevant use on demand
at any time.


\printbibliography{}

\end{document}

%%% Local Variables:
%%% mode: latex
%%% TeX-master: t
%%% End:

%  LocalWords:  SYS ISA SQL SRS llrr CDRs CDR PyCharm IDE
