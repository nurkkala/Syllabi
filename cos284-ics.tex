% Created 2016-02-02 Tue 21:42
\documentclass{article}
\usepackage[utf8]{inputenc}
\usepackage[T1]{fontenc}
\usepackage{fixltx2e}
\usepackage{graphicx}
\usepackage{grffile}
\usepackage{longtable}
\usepackage{wrapfig}
\usepackage{rotating}
\usepackage[normalem]{ulem}
\usepackage{amsmath}
\usepackage{textcomp}
\usepackage{amssymb}
\usepackage{capt-of}
\usepackage{hyperref}
\usepackage{booktabs}
\usepackage[margin=1.5in]{geometry}
\usepackage{lastpage}
\usepackage{fancyhdr}
\pagestyle{fancy}
\lhead{COS 284---Introduction to Computer Systems}
\chead{}
\rhead{Course Syllabus}
\lfoot{Spring 2016}
\cfoot{}
\rfoot{Page \thepage\ of \pageref{LastPage}}
\renewcommand{\headrulewidth}{0.4pt}
\renewcommand{\footrulewidth}{0.4pt}
\renewcommand\maketitle\relax
\date{Spring, 2016}
\title{COS 284---Introduction to Computer Systems}
\hypersetup{
 pdfauthor={Tom Nurkkala},
 pdftitle={COS 284---Introduction to Computer Systems},
 pdfkeywords={},
 pdfsubject={},
 pdfcreator={Emacs 24.5.1 (Org mode 8.3.3)}, 
 pdflang={English}}
\begin{document}

\maketitle
\section{Instructor}
\label{sec:orgheadline1}
\textbf{Dr. Tom Nurkkala}\\
Associate Professor, Computer Science and Engineering\\
Director, Taylor Center for Missions Computing\\
Office: Euler Science Complex 211\\
E-Mail: \texttt{tnurkkala@cse.taylor.edu}\\
Phone: 8-5163\\
Hours: MW~2:00-4:00, R~1:00-2:00, or by appointment\\

\section{Credit}
\label{sec:orgheadline2}
This course is based on the Introduction to Computer Systems course at Carnegie Mellon
University (CMU). Professors Randal Bryant and David O’Hallaron, who developed the course
and wrote the textbook (Section 5), have generously made available course material
(including notes, syllabi, laboratory exercises, and lecture slides) for use at other
institutions. I make considerable use of their excellent materials throughout the course
(including in this document!) and wish to credit and thank them.

I have also drawn liberally from previous offerings of this course at Taylor University,
and wish to thank Professor Jonathan Geisler for his kind assistance.

\section{Course Overview}
\label{sec:orgheadline3}
The aim of the course is to help you become a better programmer by teaching you the basic
concepts underlying all computer systems. I want you to learn what really happens when
your programs run, so that when things go wrong (as they always do) you will have the
intellectual tools to solve the problem.

Why do you need to understand computer systems if you do all of your programming in
high-level languages? In most of computer science, we're pushed to make abstractions and
stay within their frameworks. But, any abstraction ignores effects that can become
critical. As an analogy, Newtonian mechanics ignores relativistic effects. The Newtonian
abstraction is completely appropriate for bodies moving at less than \(0.1c\), but higher
speeds require working at a greater level of detail.

The following ``realities'' are some of the major areas where the abstractions you've
learned in previous classes break down:
\begin{enumerate}
\item \emph{An \texttt{int} isn't an integer; a \texttt{float} isn't a real}.
Our finite representations of numbers have significant limitations,
and because of these limitations
we sometimes have to think in terms of bit-level representations.
\item \emph{You've got to know assembly language}.
Even if you never write programs in assembly,
the behavior of a program sometimes cannot be understood
based purely on the abstraction of a high-level language.
Furthermore, understanding the effects of bugs
requires familiarity with the machine-level model.
\item \emph{Memory matters}.
Computer memory is not unbounded.
It must be allocated and managed.
Memory referencing errors are especially pernicious.
An erroneous updating of one object can cause a change
in some logically unrelated object.
Also, the combination of caching and virtual memory
provides the functionality of a uniform unbounded address space,
but not the performance.
\item \emph{There is more to performance than asymptotic complexity}.
Constant factors also matter.
There are systematic ways to evaluate and improve program performance.
\item \emph{Computers do more than execute instructions}.
They also need to get data in and out
and they interact with other systems over networks.
\end{enumerate}

By the end of the course, you will understand these realities in some detail.
As a result, you will have learned skills and knowledge
that will help you throughout your education and career as a computer scientist.

\section{Learning Objectives}
\label{sec:orgheadline4}

By the end of the semester, you should be able to:

\begin{enumerate}
\item Describe how a computer represents integers internally.
\item Convert twos complement binary values to decimal.
\item Convert binary to decimal and hexadecimal.
\item Understand how a computer represents fractional values.
\item Convert a fractional decimal value to its IEEE single- or double-precision representation.
\item Read x86 assembly code.
\item Write simple x86 assembly functions.
\item Describe how a computer executes a function call
including parameters, return values, and the stack.
\item Describe how compilers convert C code to x86 assembly.
\item Explain how various C data structures are accessed in x86 assembly.
\item Describe the motivation for a memory hierarchy.
\item Explain how a memory hierarchy works.
\item Understand how a cache is organized internally.
\item Create programs that take advantage of the memory hierarchy.
\item Explain how a linker works.
\item Contrast static and dynamic linking.
\item Explain what a system call is and how system calls work.
\item Explain the fork system call.
\item Explain the exec family of system calls.
\item Describe how hardware and the operating system cooperate to deal with exceptional
situations.
\item Measure the performance of an application accurately.
\item Improve the performance of an application based on good measurement techniques.
\item Understand virtual memory and how it works.
\item Explain the benefits of virtual memory.
\item Explain the role of the translation look-aside buffer.
\item Describe the virtual memory organization
of a process running with Linux on an x86 processor.
\item Explain how a dynamic memory allocator works.
\item Be familiar with garbage collection.
\item Be familiar with common memory bugs and how to avoid them.
\item Explain what concurrent computation is and how to achieve it.
\item Contrast process-based concurrency with I/O multiplexing-based concurrency.
\item Describe why synchronization is necessary.
\item Be familiar with how to achieve synchronization.
\end{enumerate}

\section{Texts}
\label{sec:orgheadline5}
The text by Bryant and O'Halaron~\cite{cs:app3} is required.
We will refer to it as \texttt{CS:APP}.
You may be able to make use of the previous edition~\cite{cs:app2} of the text,
but I will not guarantee that all the material in the course will be
covered by that edition.
You are \emph{encouraged} to purchase---or have readily available---the
updated edition of ``K\&R,'' the classic reference on C.\cite{k&r}

It is \emph{very important} that you read carefully
the assigned readings in the \texttt{CS:APP} book.
You will get the most benefit if you read assigned passages
\emph{before} they are covered in class.
A detailed guide to reading assignments can be found in the course schedule.

The textbook includes many practice problems throughout the body of each chapter. These
are straightforward exercises that help you understand the material you have just read by
using it \emph{immediately}. For self-study, solutions for all practice problems appear at the
end of each chapter.

In short, the best way to use the textbook to enhance your learning is as follows:
\begin{enumerate}
\item Read the assigned readings \emph{before} the corresponding class.
\item Work the practice problems \emph{immediately} as you encounter them in the text.
\item Check your work—and your understanding—with the answer key.
\end{enumerate}

\section{Topics}
\label{sec:orgheadline6}
These are the topics that could potentially be covered in the course.
It is unlikely that we will have time to cover all this material.
\begin{itemize}
\item Course Overview
\item Bits and Bytes
\begin{itemize}
\item Basics
\item Integers
\item Floating Point
\end{itemize}
\item Machine Programming
\begin{itemize}
\item Basics
\item Control
\item Procedures
\item Data
\item Advanced
\end{itemize}
\item Program Optimization
\item Memory Hierarchy
\item Cache Memory
\item Linking
\item Exceptional Control Flow
\begin{itemize}
\item Exceptions and Processes
\item Signals and Non-Local Jumps
\end{itemize}
\item Virtual Memory
\begin{itemize}
\item Concepts
\item Systems
\end{itemize}
\item Dynamic Memory
\begin{itemize}
\item Basic
\item Advanced
\end{itemize}
\item Internetworking
\begin{itemize}
\item Network Programming
\item Web Servers
\end{itemize}
\item Concurrent Programming
\begin{itemize}
\item Sychronization
\end{itemize}
\end{itemize}
\section{Labs}
\label{sec:orgheadline7}

The labs come directly from the textbook authors. They have a very nice system set up so
that you can test and submit all your programs online and will already know how you are
doing on the assignment prior to submission. This should make it very easy to know when
you have a correct solution and when you need to keep working.

Please do not look for solutions online. Doing so constitutes cheating.  The authors have
introduced randomness into the assignments so solutions posted by others may be wrong
anyway.

You will use Linux for all lab work. For some labs, you will be given a binary that works
on the CSE machines, but should probably work on any recent vintage of Linux. For others,
you are required to develop code for a Linux box. You may use any machine to develop, but
\emph{it must run correctly on the CSE machines}. You must check your work on the machines in the
laboratories before submitting it.

Most low-level systems code is written in~C, and you will be required to do the same. You
will also be required to read, understand, generate, or otherwise fiddle with x86 assembly
on a Linux box.

\section{Evaluation}
\label{sec:orgheadline8}
The grading scheme for the course will be announced in the next few days.

\section{Moodle}
\label{sec:orgheadline9}

The Computer Science and Engineering department uses Moodle as our Learning Management
System. The URL for Moodle is \url{https://moodle.cse.taylor.edu}. To sign on to the course site
for the first time, you will need an enrollment key. The key for this course is
\texttt{nerds4christ}.

You are responsible for checking Moodle regularly to keep up with assignment due dates and
other announcements posted to the site. For due dates, the Moodle calendar is your friend.
\section{Classroom Expectations}
\label{sec:orgheadline13}

Following are my expectations about classroom conduct.

\subsection{Attendance}
\label{sec:orgheadline10}

Attendance is required. I will be in class each day, and I expect you to be there also. I
will log who attends class.

In general, I am very understanding about students who must miss class due to a sanctioned
Taylor activity, medical appointment, job interview, family emergency, and the like. If
possible, let me know in advance that you will not be in class; I will work with you to
arrange make-up instruction, homework, exams, etc.

\subsection{Conduct}
\label{sec:orgheadline11}

I expect you to be prepared, awake, aware, and participatory during class. I will not
hesitate to ask you to stand or move if you are distracted or sleepy.

I expect you to join in discussions, respond to questions from me and from your
colleagues, and ask questions of me. I expect you to hold my feet to the fire if I am
being unclear, unkind, or contradictory.

\subsection{Gizmos}
\label{sec:orgheadline12}

You may not use a laptop, tablet, or similar device to check e-mail, engage in social
networking, surf the web, or any other activity not directly relevant to current classroom
activity.

If you use an electronic gizmo during class for legitimate academic purposes (e.g., note
taking), be prepared to demonstrate relevant use on demand at any time.

\section{Academic Integrity}
\label{sec:orgheadline16}

As a student at an institution whose goal is to honor Christ in all that it does, i expect
you to uphold the strictest standards of academic integrity. You must do your own work,
cite others when you present their work, and never misrepresent your academic performance
in any way. Violation of these standards stains the reputations of you as a student,
Taylor as an institution, and Jesus as our Lord. Such a violation may result in your
failing the course and other disciplinary action by the University. Refer to the Taylor
catalog for the official statement of these ideas.

\subsection{What Constitutes Cheating?}
\label{sec:orgheadline14}

For purposes of this course, the following are \emph{non-exhaustive} examples of violations of
academic integrity.

\begin{enumerate}
\item Sharing code or other electronic files by copying, retyping, looking at, or supplying a
copy of a file from this or a previous semester. Be sure to store your work in
protected directories, and screen lock or log off a lab machine to prevent others from
copying your work without your explicit assistance.
\item Sharing written assignments or exams by looking at, copying, or supplying an assignment
or exam.
\item Using other's code. Using code from this or previous offerings of the class, from
courses at other institutions, or from any other source (e.g., software found on the
Internet).
\item Looking at other's code. Although mentioned above, it bears repeating. Looking at other
students' code or allowing others to look at yours is cheating. There is no notion of
looking “too much,” since no looking is allowed at all.
\end{enumerate}

\subsection{What Does Not Constitute Cheating?}
\label{sec:orgheadline15}

In contrast, the following are non-exhaustive examples of activities that do not violate
academic integrity.

\begin{enumerate}
\item Clarifying ambiguities or vague points in class handouts or textbooks.
\item Helping others use the computer systems, networks, compilers, debuggers, profilers, or
other system facilities.
\item Helping others with high-level design issues.
\item Helping others with high-level (not code-based) debugging.
\item Using code from the \texttt{CS:APP} website or from the class web pages.
\end{enumerate}

Be sure to store your work in protected directories, and log off when you leave an open
cluster, to prevent others from copying your work without your explicit assistance.

\bibliographystyle{plain}
\bibliography{courses}
\end{document}
%%% Local Variables:
%%% mode: latex
%%% TeX-master: t
%%% End:
