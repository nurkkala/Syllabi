\documentclass[11pt]{article}

\usepackage{framed}
\usepackage{xcolor}

\newcommand{\syllabuslayout}[3]{
  \usepackage[margin=1.5in]{geometry}
  \usepackage{lastpage}

  \usepackage{fancyhdr}
  \renewcommand{\headrulewidth}{0.4pt}
  \renewcommand{\footrulewidth}{0.4pt}
  \pagestyle{fancy}

  \lhead{Course Syllabus}
  \chead{}
  \rhead{Page \thepage\ of \pageref{LastPage}}
  \lfoot{#1}
  \cfoot{#2}
  \rfoot{#3~\the\year}
}

\newcommand{\titleblock}[2]{
  \begin{center}
    \color{purple}
    {\bfseries
      {\Large #1}\\[4pt]
      {\Large #2}}\\[12pt]
    Department of Computer Science and Engineering\\
    Taylor University
  \end{center}
}

\newcommand{\logistics}[4]{
  \begin{center}
    #1 Credit Hours\\
    #2 $\bullet$\ #3\\[1em]
    Final Exam: #4
  \end{center}
}

\newcommand{\lastupdated}{
\vfill
\begin{flushright}
  \footnotesize
  \textit{This document last updated \today}.
\end{flushright}}

\newenvironment{catalogentry}[1]
{\section{Description}
  \textbf{Prerequisites}: #1\\

  \noindent
  \textbf{From the catalog}:}
{}

\newcommand{\safari}{O'Reilly Learning}
\syllabuslayout{COS~433}{Missions Computing---Senior Project}{Interterm}

\usepackage[colorlinks=true,allcolors=blue]{hyperref}
\usepackage{booktabs}
\usepackage{comment}

% \usepackage{biblatex}
% \addbibresource{courses.bib}

\begin{document}

\section{Instructor}
\begin{center}
  \begin{tabular}{rll}
    \toprule
    \multicolumn{3}{l}{\textbf{Dr.\ Tom Nurkkala}}                            \\
    \multicolumn{3}{l}{Associate Professor, Computer Science and Engineering} \\
    \multicolumn{3}{l}{Director, Center for Missions Computing}               \\
    \midrule
    Office & \multicolumn{2}{l}{Euler Science Complex 211}                    \\
    Email  & \multicolumn{2}{l}{\texttt{tnurkkala@cse.taylor.edu}}            \\
    Phone  & \multicolumn{2}{l}{765/998-5163}                                 \\
    Hours  & Mon, Wed & 2:00--3:00                                            \\
           & Tue, Thu & 10:00--12:00                                          \\
           & \multicolumn{2}{l}{\emph{Or by appointment}}                     \\
    \bottomrule
  \end{tabular}
\end{center}

%%% Local Variables:
%%% mode: latex
%%% End:

%  LocalWords:  rrl


\section{Introduction}

Missions Computing---Senior Project
(just \emph{Missions Computing} in the remainder of this document)
is a course primarily intended
for students in Computer Science and Engineering
who wish to engage in service-learning activities---%
particularly software development---%
with a partner in the international missions community.
\begin{itemize}
\item
  Admission to the course is by instructor permission.
\item
  This course carries Cross Cultural (CC)
  and Speaking (SP) credit.
\end{itemize}

\section{Objectives}

The Missions Computing course engages you directly
in missions computing in an international context
where missions computing systems are created or used (or both).
In particular, you will:
\begin{enumerate}
\item
  Apply your understanding of computer science
  and expertise in software development by
  contributing to the design, construction, enhancement, testing, deployment,
  documentation, or support of software
  employed directly by a missions partner.
\item
  Travel internationally to a missions partner location
  at which software and systems are developed, supported, or deployed
  in direct support of missions operations.
\item
  Experience the mission partner’s organization
  through engagement with missions staff and families
  in service, fellowship, prayer, and worship.
  One goal of this experience is to
  ``demystify'' missions service,
  letting you see that your knowledge, abilities, and skills
  can be employed directly in service to the gospel.
\item
  Engage the local culture of the missions partner
  through service, evangelism, and mentoring
  under the leadership of the missions partner.
\item
  Understand how God superintends
  the advancement of computing and related technology,
  making it available to Christian technologists
  to obey the Great Commission and Great Commandments.
\item
  See firsthand how software and systems
  impact and enable the work of the missions
  partner both in the office and in the field.
\item
  Understand the need throughout the missions community
  for financial and human resources
  in support of computing and related needs.
\end{enumerate}

\section{Content}

The course comprises the following key elements.

\subsection{Course Preparation}

Because the travel component of the course
qualifies as a mission trip
(more than 50\% of course activities
are directly related to missions service),
you are eligible to raise
tax-deductable support through Taylor University.
You are required to contact supporters who are willing
to provide prayer or financial support for the trip.
In addition, you will:
\begin{enumerate}
\item
  Participate in team organization and information meetings
  facilitated by the team leaders.
\item
  Read and study technical material
  relevant to the missions computing project
  as directed by the team leaders.
\end{enumerate}

\subsection{International Experience}

The majority of the course
consists of international experience serving with a missions partner.
During this time, you will:
\begin{enumerate}
\item
  Learn about the missions partner
  through orientation presentations, discussion, and activities,
  usually facilitated by staff at the missions partner.
\item
  Acclimate to the community in which you will serve
  by various means, including home stays with mission staff,
  worship services, community engagement, and other activities.
\item
  Develop relationships with missions staff
  through shared service on team pro\-jects
  and through participation in normal partner activities
  such as team devotions, Bible study, corporate prayer, and the like.
\item
  Receive training on the software and systems
  with which you will be working while in country
  with the missions partner.
  Typical topics include software engineering practice and process,
  languages and frameworks, logical modeling, revision control,
  issue tracking, continuous integration, and so forth.
\item
  Design, develop, enhance, repair, or extend software systems
  at the direction of the technical leadership at the mission.
  This activity constitutes a considerable fraction
  of your time at the mission partner
  (typically ``full time'' work hours most days).
\item
  Enjoy and appreciate cultural, social, and ethnic diversity
  through travel, dining, sightseeing, and similar activities.
\item
  Reflect on your travel and service
  through introspection, discussion, and required journaling.
\end{enumerate}

\subsection{Course Wrap-Up}

By the last day of the term, you will:
\begin{enumerate}
\item Complete, clean up, and submit your journal (Section~\ref{sec:journal}).
\item Write and submit your final experience paper (Section~\ref{sec:exp-paper}).
\end{enumerate}
Upon return to campus, you will:
\begin{enumerate}
\item Present your poster (Section~\ref{sec:poster}).
\item Attend and participate in any final debriefing meetings,
  team celebration, campus presentation, etc.
\end{enumerate}

\section{Deliverables}

You must submit the following deliverables on the last day of the course.

\subsection{Journal}
\label{sec:journal}

The journal is a daily written record
of your experience throughout the course,
including the time before, during, and after
international travel and service.
You are expected to make at least one entry per day,
but are welcome to make more than one.
Each is to be tagged with the date and location at which the entry was made.
These entries will be read and evaluated by the instructor,
but will not be shared with other team members
unless you authorize or encourage it.
Over the duration of the course,
your journal should include (but is not limited to):
\begin{enumerate}
\item
  Motivations for participation in the course
\item
  Expectations for the course prior to departure,
  including open questions that you hope to explore and answer during the course
\item
  Travel experience (to, from, and in the field)
\item
  Experience serving with the missions partner
  from technical, personal, social, and spiritual perspectives
\item
  Observations and insights into the culture(s) served during the trip
\item
  Changes in your view of culture, economics, government,
  technology, relationships, missions, theology, and spirituality
  (both in the international culture and at home)
\item
  Answers or insights into the questions you hoped to address during the course
\item
  Ways in which the experience altered, clarified, or informed
  your vocational calling as a computer scientist seeking to serve Jesus
\item
  Aspects of the course that were important, meaningful, or just plain fun
\item
  Suggestions as to how the course could be improved in the future
\end{enumerate}

\subsection{Final Experience Paper}
\label{sec:exp-paper}

You will write a paper about your personal experience during the course.
The goal of this paper is to reflect on your own experience in the course
and how you matured as a computer scientist and as a Christian.
Your paper should address at least the following questions.
\begin{enumerate}
\item
  What stood out to you as unexpected or otherwise significant
  with regard to your perception of a culture other than your own?
\item
  What insights did you gain regarding missions service in general?
\item
  What did you learn about yourself
  as it relates specifically to serving as a member of a missions computing team?
\item
  What were the most important knowledge and skills you acquired
  as it relates to your future as a computer scientist?
\item
  How did your experience speak to your vocational call as a Christ follower?
\end{enumerate}
These questions are not intended to be exhaustive.
You are encouraged to reflect in your paper
on any additional insights you gleaned from your experience.

Type your paper.
Please double space.
Use good spelling, grammar, punctuation, and structure.
Your paper should be 1,250 to 1,500 words long.
Print your paper and submit it to the course web site.

\section{Presentations and Poster}
\label{sec:poster}

This course carries SP (speaking) credit.
Consequently,
you will make two presentations.

First,
you will present the results of your project work
to our missions partner.
In most cases,
this will be a group presentation.
If the project involved multiple teams,
each will present on its own work.

Second,
you will prepare a poster
that captures the work you did on the project,
and you will present the poster
to the scholarly community at Taylor.
As with the presentation to our missions partner,
the poster may be prepared and presented
as a team (perhaps multiple teams).
The poster presentation will be scheduled
on campus
early in the semester following
the overseas experience.

\section{Participation}

Barring sickness or injury, you are expected
to attend all the meetings, activities, and team project work
throughout the course
(before, during, and after international travel).

\section{Evaluation}

Refer to my Periodic Table of the Grades (on Moodle)
for my standard grading scheme.
I reserve the right to award a higher grade
than strictly earned; outstanding contributions,
leadership, and participation figure prominently in such decisions.
Course criteria contribute to your grade according to the following table.
\begin{center}
  \begin{tabular}{lr}
    \toprule
    Criterion                   & Weight \\
    \midrule
    Teamwork and participation  & 10\%   \\
    Mature, Christlike behavior & 10\%   \\
    \midrule
    Project contributions       & 30\%   \\
    \midrule
    Journal                     & 10\%   \\
    Final experience paper      & 10\%   \\
    Customer Presentation       & 15\%   \\
    Poster/Presentation         & 15\%   \\
    \midrule
                                & 100\%  \\
    \bottomrule
  \end{tabular}
\end{center}

% \section{Course Expectations}
% Following are my expectations regarding the course.

\subsection{Attendance}
\label{sec:attendance}

You are required to attend all class sessions.
I will be in class each day, and I expect you to be there also.

In general, I am very understanding about students who must miss class
due to a sanctioned Taylor activity, medical appointment, job interview,
family emergency, and the like.
If possible, let me know in advance that you will not be in class;
I will work with you to arrange make-up instruction, homework, exams, etc.

\subsection{Late Work}

All course assignments will include an unambiguous due date.
Usually, assignments are due at the beginning of class on the due date.
If there are multiple sections of a class,
the assignment is due at the beginning of the earliest such section.
Barring exceptional circumstances like those mentioned in section~\ref{sec:attendance},
I expect your work to be submitted \emph{on the due date}.
Late work will \emph{not} be accepted.

This policy on late work is intended to prepare you
for real-world experience after graduation.
In the marketplace,
late work is not merely an inconvenience.
Missing a deadline may
alienate your customer,
upset your manager,
ruin your project,
or terminate your employment!
\emph{Now} is the time to learn the self discipline and time management skills
required to complete your work when it is due.

\subsection{Conduct}

I expect you to be prepared, awake, aware, and participatory during class. I will not
hesitate to ask you to stand or move if you are distracted or sleepy.

I expect you to join in discussions, respond to questions from me and from your
colleagues, and ask questions of me. I expect you to hold my feet to the fire if I am
being unclear, unkind, or contradictory.

\subsection{Gizmos}

You may not use a laptop, tablet, or similar device to check e-mail, engage in social
networking, surf the web, or any other activity not directly relevant
to current classroom activity.
If you use an electronic gizmo during class for legitimate academic purposes
(e.g., note taking), be prepared to demonstrate relevant use on demand
at any time.

\section{Moodle}

The Computer Science and Engineering department uses Moodle
as our Learning Management System.
The URL for Moodle is \url{https://moodle.cse.taylor.edu}.
To sign on to the course site for the first time,
you will need an enrollment key.
The key for this course is
\texttt{nerds4christ}.

You are responsible for checking Moodle regularly
to keep up with assignment due dates and other announcements.
For due dates, \emph{the Moodle calendar is your friend}.

\section{Slack}

This course will use Slack
for informal communication,
Q\&A,
last minute announcements,
jokes,
and the like.
You are \emph{strongly} encouraged to join the conversation.

Find the \emph{TU CSE Student} slack team at
\url{tucsestudents.slack.com}.
Look there for a \emph{channel}
dedicated to the course.

\section{Academic Integrity}

As a student at an institution whose goal is to honor Christ in all that it does,
I expect you to uphold the strictest standards of academic integrity.
You must do your own work,
cite others when you present their work,
and never misrepresent your academic performance in any way.
Violation of these standards stains the reputations of you as a student,
Taylor as an institution,
and Jesus as our Lord.

Every assignment should indicate clearly
that it is either:
\begin{itemize}
\item An \textbf{individual} assignment,
  to be done \emph{entirely by you},
  without any direct participation from other students.
\item A \textbf{group} assignment, to be done \emph{collectively with a group}
\end{itemize}
Unless otherwise stated,
assignments are \textbf{individual} assignments.

\begin{framed}
Note that you are \emph{always} welcome
to get help from the instructor.
\end{framed}

A violation of academic integrity may result in your failing the course
and other disciplinary action by the University.
Refer to the Taylor catalog for the official statement of these ideas.

%  LocalWords:  christ

%%% Local Variables:
%%% mode: latex
%%% TeX-master: "sys394-isd"
%%% End:


% \printbibliography{}

\end{document}

%%% Local Variables:
%%% mode: latex
%%% TeX-master: t
%%% End:

% LocalWords:  ISD Kepner Tregoe SRS ISA llllll christ multi OpenMP MPI GPUs
% LocalWords:  GPU Barlas Cheng et al Storti Yurtoglu jects lr
